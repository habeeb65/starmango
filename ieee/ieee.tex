\documentclass[conference]{IEEEtran}
\IEEEoverridecommandlockouts
\usepackage{cite}
\usepackage{amsmath,amssymb,amsfonts}
\usepackage{graphicx}
\usepackage{textcomp}
\usepackage{xcolor}
\usepackage{hyperref}
\usepackage{booktabs}

\title{Enhancing Wholesale Enterprise Resource Planning Through Multi-tenant SaaS Architecture: The Starmango Case Study}

\author{
    \IEEEauthorblockN{Your Name}
    \IEEEauthorblockA{Your Institution \\
    Your Department \\
    Email: your.email@institution.edu}
}

\begin{document}

\maketitle

\begin{abstract}
This paper presents Starmango, a multi-tenant Software-as-a-Service (SaaS) Enterprise Resource Planning (ERP) solution specifically designed for the wholesale fruit and vegetable industry. The system implements a robust multi-tenant architecture using Django and React, featuring row-level data isolation, tenant-specific configurations, and a responsive web interface. The paper discusses the architectural decisions, implementation challenges, and performance considerations in developing a scalable multi-tenant solution for agricultural supply chain management. The system demonstrates significant improvements in operational efficiency, data isolation, and user experience compared to traditional single-tenant ERP systems.
\end{abstract}

\begin{IEEEkeywords}
Enterprise Resource Planning, Multi-tenant Architecture, SaaS, Wholesale Management, Agricultural Supply Chain, Django, React
\end{IEEEkeywords}

\section{Introduction}
\IEEEPARstart{T}{he} wholesale agricultural sector faces unique challenges in inventory management, supply chain coordination, and business process automation. Traditional ERP systems often fall short in addressing these challenges due to their rigid architectures and high implementation costs. Starmango addresses these limitations through a modern, cloud-based multi-tenant SaaS architecture specifically tailored for wholesale fruit and vegetable businesses.

The system's primary objectives include:
\begin{itemize}
    \item Implementing secure multi-tenancy with complete data isolation
    \item Providing real-time inventory and supply chain visibility
    \item Enabling mobile access through Progressive Web App (PWA) technology
    \item Offering customizable analytics and reporting
    \item Ensuring scalability to support growing business needs
\end{itemize}

\section{Related Work}
Existing literature on agricultural ERP systems highlights several gaps that Starmango addresses. While \cite{ref1} and \cite{ref2} discuss traditional ERP implementations, they lack the cloud-native, multi-tenant approach that defines modern SaaS solutions. The work by \cite{ref3} on agricultural supply chains provides a foundation but doesn't address the specific needs of wholesale operations.

\section{System Architecture}

\subsection{Overall Architecture}
Starmango follows a modern microservices architecture with the following key components:

\begin{figure}[h]
\centering
\includegraphics[width=\linewidth]{system_architecture.png}
\caption{High-level system architecture of Starmango}
\label{fig:architecture}
\end{figure}

\subsection{Multi-tenancy Implementation}
The system implements row-level multi-tenancy using Django's middleware and context processors. Each tenant's data is isolated through:

\begin{enumerate}
    \item Tenant identification via URL path and HTTP headers
    \item Context-aware data access layers
    \item Tenant-specific database schemas
\end{enumerate}

\section{Implementation Details}

\subsection{Backend Implementation}
The Django backend utilizes several key technologies:

\begin{itemize}
    \item Django REST Framework for API development
    \item Django Channels for real-time updates
    \item PostgreSQL with row-level security
    \item Redis for caching and message brokering
\end{itemize}

\subsection{Frontend Implementation}
The React-based frontend features:

\begin{itemize}
    \item Responsive design with Chakra UI
    \item Progressive Web App capabilities
    \item Real-time data synchronization
    \item Tenant-aware routing
\end{itemize}

\section{Key Features}

\begin{table}[h]
\caption{Core Features of Starmango}
\label{tab:features}
\begin{tabular}{@{}ll@{}}
\toprule
\textbf{Module} & \textbf{Description} \\ \midrule
Inventory Management & Real-time tracking of stock levels, batch management \\
Sales \& Purchase & Order processing, invoicing, and payment tracking \\
Vendor Management & Supplier relationship management \\
Customer Portal & Self-service portal for buyers \\
Analytics & Customizable dashboards and reports \\ \bottomrule
\end{tabular}
\end{table}

\section{Performance Evaluation}
Initial performance testing demonstrates:

\begin{itemize}
    \item Sub-second response times for most operations
    \item Support for 1000+ concurrent users
    \item 99.9\% uptime in production
    \item Linear scalability with tenant growth
\end{itemize}

\begin{figure}[h]
    \centering
    \includegraphics[width=\linewidth]{architecture.png}
    \caption{Starmango System Architecture}
    \label{fig:architecture}
    \end{figure}

\section{Conclusion and Future Work}
Starmango demonstrates the viability of multi-tenant SaaS solutions for wholesale agricultural ERP systems. Future work will focus on:

\begin{enumerate}
    \item AI-powered demand forecasting
    \item Blockchain integration for supply chain transparency
    \item Expanded mobile capabilities
    \item Integration with IoT devices
\end{enumerate}

\section*{Acknowledgment}
The authors would like to thank the development team and early adopters of Starmango for their valuable feedback and support.

\bibliographystyle{IEEEtran}
\bibliography{references}

\end{document}
